
% Copyright (c) 1991-2002, The Numerical ALgorithms Group Ltd.
% All rights reserved.
%
% Redistribution and use in source and binary forms, with or without
% modification, are permitted provided that the following conditions are
% met:
%
%     - Redistributions of source code must retain the above copyright
%       notice, this list of conditions and the following disclaimer.
%
%     - Redistributions in binary form must reproduce the above copyright
%       notice, this list of conditions and the following disclaimer in
%       the documentation and/or other materials provided with the
%       distribution.
%
%     - Neither the name of The Numerical ALgorithms Group Ltd. nor the
%       names of its contributors may be used to endorse or promote products
%       derived from this software without specific prior written permission.
%
% THIS SOFTWARE IS PROVIDED BY THE COPYRIGHT HOLDERS AND CONTRIBUTORS "AS
% IS" AND ANY EXPRESS OR IMPLIED WARRANTIES, INCLUDING, BUT NOT LIMITED
% TO, THE IMPLIED WARRANTIES OF MERCHANTABILITY AND FITNESS FOR A
% PARTICULAR PURPOSE ARE DISCLAIMED. IN NO EVENT SHALL THE COPYRIGHT OWNER
% OR CONTRIBUTORS BE LIABLE FOR ANY DIRECT, INDIRECT, INCIDENTAL, SPECIAL,
% EXEMPLARY, OR CONSEQUENTIAL DAMAGES (INCLUDING, BUT NOT LIMITED TO,
% PROCUREMENT OF SUBSTITUTE GOODS OR SERVICES-- LOSS OF USE, DATA, OR
% PROFITS-- OR BUSINESS INTERRUPTION) HOWEVER CAUSED AND ON ANY THEORY OF
% LIABILITY, WHETHER IN CONTRACT, STRICT LIABILITY, OR TORT (INCLUDING
% NEGLIGENCE OR OTHERWISE) ARISING IN ANY WAY OUT OF THE USE OF THIS
% SOFTWARE, EVEN IF ADVISED OF THE POSSIBILITY OF SUCH DAMAGE.


\def\Fortran{FORTRAN}

\long\def\mainPerson#1#2{{\baselineskip 10pt%
  \par\vskip 3pt\hangafter=1 \hangindent=2pc \noindent {{\bf #1}} {#2}\par}}
\def\person#1#2{{{#1}{#2}}}

\schapter{Contributors}

{\advance\parskip by -1pt
%
The design and development of \Language{} was led by the Symbolic
Computation Group of the Mathematical Sciences Department, IBM Thomas
J. Watson Research Center, Yorktown Heights, New York.
The current implementation of \Language{} is the product of many people.
The primary contributors are:
%\vskip \baselineskip \vskip -4pt
\mainPerson{Richard D. Jenks}{(IBM, Yorktown) received a Ph.D.
from the University of Illinois and was a principal architect of
the {\bf Scratchpad} computer algebra system (1971).
In 1977, Jenks initiated the \Language{} effort with the design of
MODLISP, inspired by earlier work with R\"udiger Loos (T\"ubingen),
James Griesmer (IBM, Yorktown), and David Y. Y. Yun (Hawaii).
Joint work with David R. Barton (Berkeley, California) and James
Davenport led to the design and implementation of
prototypes and the concept of categories (1980).
More recently, Jenks led the effort on user interface
software for \Language{}.}

\mainPerson{Barry M. Trager}{(IBM, Yorktown) received a Ph.D. from
MIT while working in the {\bf MACSYMA} computer algebra group.
Trager's thesis laid the groundwork for a complete theory for
closed-form integration of elementary functions and its
implementation in \Language{}.
Trager and Richard Jenks are responsible for the original abstract
datatype design and implementation of the
programming language with
its current MODLISP-based compiler and run-time system.
Trager is also responsible for the overall design of the current
\Language{} library and for the implementation of many of its
components.}

\mainPerson{Stephen M. Watt}{(IBM, Yorktown) received a Ph.D. from
the University of Waterloo and is one of the original authors of the {\bf
Maple} computer algebra system.
Since joining IBM in 1984, he has made central contributions to the
\Language{} language and system design, as well as numerous contributions
to the library.
He is the principal architect of the new \Language{} compiler,
planned for Release 2.
}

\allowbreak
\mainPerson{Robert S. Sutor}{(IBM, Yorktown) received a Ph.D. in
mathematics from Princeton University and has been involved
with the design and implementation of the system interpreter,
system commands, and documentation since 1984.
%From 1988-1991, he was on academic leave from IBM to the
%Department of Mathematics at Princeton University in order to
%complete a Ph.D. under Nicholas M. Katz.
Sutor's contributions to the \Language{} library include factored
objects, partial fractions, and the original implementation of finite
field extensions.
Recently, he has devised technology for producing automatic
hard-copy and on-line documentation from single source files.}

\mainPerson{Scott C. Morrison}{(IBM, Yorktown) received an
M.S. from the University of California, Berkeley, and is a principal
person responsible for the design and implementation of the
\Language{} interface, including the interpreter, \HyperName{},
and applications of the computer graphics system.}

\mainPerson{Manuel Bronstein}{(ETH, Z\"urich) received a Ph.D. in
mathematics from the University of California, Berkeley,
completing the theoretical work on closed-form integration by
Barry Trager.
Bronstein designed and implemented the algebraic structures and
algorithms in the \Language{} library for integration, closed form
solution of differential equations, operator algebras, and
manipulation of top-level mathematical expressions.
He also designed (with Richard Jenks) and implemented the current
pattern match facility for \Language{}.}

\mainPerson{William H. Burge}{(IBM, Yorktown) received a Ph.D.
from Cambridge University, implemented the \Language{} parser,
designed (with Stephen Watt) and implemented the stream
and power series structures, and numerous algebraic facilities
including those for data structures, \allowbreak power series, and
combinatorics.}

\mainPerson{Timothy P. Daly}{(IBM, Yorktown) is pursuing a Ph.D. in
computer science at Brooklyn Polytechnic Institute
and is responsible for porting, testing,
performance, and system support work for \Language{}.}

\mainPerson{James Davenport}{(Bath) received a Ph.D. from Cambridge
University, is the
author of several computer algebra textbooks, and has long
recognized the need for \Language{}'s generality for computer
algebra.
He was involved with the early prototype design of system internals
and the original category hierarchy for \Language{} (with David
R. Barton).
More recently, Davenport and Barry Trager designed the algebraic
category hierarchy currently used in \Language{}.
Davenport is Hebron and Medlock Professor of Information
Technology at Bath University.}

\mainPerson{Michael Dewar}{(Bath) received a Ph.D.
from the University of Bath for his work on the IRENA system
(an interface between the {\bf REDUCE} computer algebra system and the NAG Library
of numerical subprograms), and work on interfacing algebraic and numerical
systems in general.  He has contributed code to produce \Fortran{} output from
\Language{}, and is currently developing a comprehensive foreign language
interface and a link to the NAG Library for release 2 of \Language{}.
}

\mainPerson{Albrecht Fortenbacher}{(IBM Scientific Center,
Heidelberg) received a doctorate from the University of Karlsruhe
and is a designer and implementer of the type-inferencing code
in the \Language{} interpreter.
The result of research by Fortenbacher on type coercion by
rewrite rules will soon be incorporated into \Language{}.}

\mainPerson{Patrizia Gianni}{(Pisa) received a Laurea in mathematics from
the University of Pisa and is the prime author of the polynomial
and rational function component of the \Language{} library.
Her contributions include algorithms for greatest common divisors, factorization,
ideals, Gr\"obner bases, solutions of polynomial systems, and
linear algebra.
She is currently Associate Professor of Mathematics at the University of
Pisa.}

\mainPerson{Johannes Grabmeier}{(IBM Scientific Center, Heidelberg)
received a Ph.D. from University Bayreuth (Bavaria) and is
responsible for many \Language{} packages, including those
for representation theory (with Holger
Gollan (Essen)), permutation groups (with Gerhard
Schneider (Essen)), finite fields (with Alfred
Scheerhorn), and non-associative algebra (with
Robert Wisbauer (D\"usseldorf)).}

\mainPerson{Larry Lambe}{received a Ph.D.
from the University of Illinois (Chicago) and has been using
\Language{} for research in homological algebra.
Lambe contributed facilities for Lie ring and exterior algebra
calculations and has worked with Scott Morrison on various
graphics applications.}

\mainPerson{Michael Monagan}{(ETH, Z\"urich) received a Ph.D.
from the University of Waterloo and is a principal contributor to the
{\bf Maple} computer algebra system. He designed and implemented
the category hierarchy and domains for data structures (with Stephen Watt),
multi-precision floating point arithmetic, code for polynomials
modulo a prime, and also
worked on the new compiler.}

\mainPerson{William Sit}{(CCNY) received a Ph.D.
from Columbia University. He has been using \Language{} for research in
differential algebra, and contributed operations for differential polynomials
(with Manuel Bronstein).}

\mainPerson{Jonathan M. Steinbach}{(IBM, Yorktown) received a B.A.
degree from Ohio State University and has responsibility for the
\Language{} computer graphics facility.
He has modified and extended this facility from
the original design by Jim Wen.
Steinbach is currently involved in the new compiler effort.}

\mainPerson{Jim Wen,}{a graduate student in computer graphics at Brown
University, designed and implemented the original computer
graphics system for \Language{} with pop-up control panels for
interactive manipulation of graphic objects.}
\newpage
\mainPerson{Clifton J. Williamson}{(Cal Poly) received a Ph.D. in Mathematics
from the University of California, Berkeley.
He implemented the power series (with William Burge and Stephen Watt), matrix,
and limit
facilities in the library and made numerous contributions to the
\HyperName{} documentation and algebraic side of the computer graphics
facility. Williamson is currently an Assistant Professor of Mathematics
at California Polytechnic State University, San Luis Obispo.}

{\sloppy
Contributions to the current \Language{} system were also made by:
\person{Yurij Baransky}{ (IBM Research, Yorktown)},
\person{David R. Barton}{},
\person{Bruce Char}{ (Drexel)},
\person{Korrinn Fu}{},
\person{R\"udiger Gebauer}{},
\person{Holger Gollan}{ (Essen)},
\person{Steven J. Gortler}{},
\person{Michael Lucks}{},
\person{Victor Miller}{ (IBM Research, Yorktown)},
\person{C. Andrew Neff}{ (IBM Research, Yorktown)},
\person{H. Michael M\"oller}{ (Hagen)},
\person{Simon Robinson}{},
\person{Gerhard Schneider}{ (Essen)},
\person{Thorsten Werther}{ (Bonn)},
\person{John M. Wiley}{},
\person{Waldemar Wiwianka}{ (Paderborn)},
\person{David Y. Y. Yun}{ (Hawaii)}.

}

\noindent
Other group members, visitors and contributors to \Language{} include
Richard Anderson,
George Andrews,
David R. Barton,
Alexandre Bouyer,
Martin Brock,
Florian Bundschuh,
Cheekai Chin,
David V. Chudnovsky,
Gregory V. Chudnovsky,
Josh Cohen,
Gary Cornell,
Jean Della Dora,
Claire DiCrescendo,
Dominique Duval,
Lars Erickson,
Timothy Freeman,
Marc Gaetano,
Vladimir A. Grinberg,
Florian Bundschuh,
Oswald Gschnitzer,
Klaus Kusche,
Bernhard Kutzler,
Mohammed Mobarak,
Julian A. Padget,
Michael Rothstein,
Alfred Scheerhorn,
William F. Schelter,
Martin Sch\"onert,
Fritz Schwarz,
Christine J. Sundaresan,
Moss E. Sweedler,
Themos T. Tsikas,
Berhard Wall,
Robert Wisbauer, and
Knut Wolf.

\vskip \baselineskip

\noindent
This book has contributions from several people in addition to its
principal authors.
Scott Morrison is responsible for the computer graphics gallery
and the programs in \appxref{ugAppGraphics}.
Jonathan Steinbach wrote the original version of Chapter 7.
Michael Dewar contributed material on the FORTRAN interface in Chapter 4.
Manuel Bronstein, Clifton Williamson,
Patricia Gianni, Johannes Grabmeier, and Barry
Trager, and Stephen Watt contributed to Chapters 8 and 9 and Appendix E.
William Burge,
Timothy Daly, Larry Lambe, and William Sit contributed material to Chapter 9.

\vskip \baselineskip

%% Ken said to not include Ken, Ruediger and Howard in the list
\noindent
The authors would like to thank the production staff at Springer-Verlag
for their guidance in the preparation of this book, and
Jean K. Rivlin of IBM Yorktown Heights for her assistance in producing the
camera-ready copy.
Also, thanks to
Robert F. Caviness,
James H. Davenport,
Sam Dooley,
Richard J. Fateman,
Stuart I. Feldman,
Stephen J. Hague,
John A. Nelder,
Eugene J. Surowitz,
Themos T. Tsikas,
James W. Thatcher, and
Richard E. Zippel
for their constructive suggestions on drafts of the book.%
}

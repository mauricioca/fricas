
\head{schapter}{Introduction to \Language{}}{ugNewIntro}

\texht{\pagenumbering{arabic}}{}


Welcome to the world of \Language{}.
We call \Language{} a scientific computation system:
a self-contained toolbox designed to meet
your scientific programming needs,
from symbolics, to numerics, to graphics.

This introduction is a quick overview of what \Language{} offers.

\pseudoSection{Symbolic computation}

\Language{} provides a wide range of
simple commands for
symbolic mathematical problem solving.
Do you need to solve an equation,
to expand a series,
or to obtain an integral?
If so, just ask \Language{} to do it.

%
\xtcLong{
Integrate \texht{${{1\over{(x^3 \  {(a+b x)}^{1/3})}}}$}{} with \break
respect to \smath{x}.
}{
\spadcommand{integrate(1/(x**3 * (a+b*x)**(1/3)),x)}
}{
\[
{\left(
\begin{array}{@{}l}
\displaystyle
-{2 \ {b^2}\ {x^2}\ {\sqrt{3}}\ {\log \left({{{\root{3}\of{a}}\ {{\root{3}\of{{b
\  x}+ a}}^2}}+{{{\root{3}\of{a}}^2}\ {\root{3}\of{{b \  x}+
a}}}+ a}\right)}}+
\\
\\
\displaystyle
{4 \ {b^2}\ {x^2}\ {\sqrt{3}}\ {\log \left({{{{\root{3}\of{a}}^
2}\ {\root{3}\of{{b \  x}+ a}}}- a}\right)}}+
\\
\\
\displaystyle
{{12}\ {b^2}\ {x^2}\ {\arctan \left({{{2 \ {\sqrt{3}}\ {{\root{3}\of{a}}^
2}\ {\root{3}\of{{b \  x}+ a}}}+{a \ {\sqrt{3}}}}\over{3 \  a}}\right)}}+

\\
\\
\displaystyle
{{\left({{12}\  b \  x}-{9 \  a}\right)}\ {\sqrt{3}}\ {\root{3}\of{a}}\ {{\root{3}\of{{b
\  x}+ a}}^2}}
\end{array}
\right)}\over{{18}\ {a^2}\ {x^2}\ {\sqrt{3}}\ {\root{3}\of{a}}}
\]
}{1}{%
Union(Expression Integer, ...)%
}

\newpage
\Language{} provides state-of-the-art algebraic machinery to
handle your most advanced symbolic problems.
For example, \Language{}'s integrator gives you the answer
when an answer exists.
If one does not, it provides a proof that
there is no answer.
Integration is just one of a multitude of symbolic operations that
\Language{} provides.

\pseudoSection{Numeric computation}

\Language{} has a numerical library that includes operations for
linear algebra, solution of equations, and special functions.
For many of these operations, you can select any number of
floating point digits to be carried out in the computation.

%
\xtcLong{
Solve \texht{$x^{49}-49x^4+9$}{} to
49 digits \break of accuracy.
}{
\spadcommand{solve(x**49-49*x**4+9 = 0,1.e-49)}
}{
\[
\begin{array}{@{}l}
\displaystyle
\left[{x = -{0.6546536706904271136718122105095984761851224331
556}}, \: \right.
\\
\\
\displaystyle
\left.{x ={1.086921395653859508493939035954893289009213388763}},
 \: \right.
\\
\\
\displaystyle
\left.{x ={0.654653670725527173969468606613676483536148760766
1}}\right]
\end{array}
\]
}{2}{%
List Equation Polynomial Float%
}

The output of a computation can be converted to FORTRAN to be used
in a later numerical computation.
Besides floating point numbers, \Language{} provides literally
dozens of kinds of numbers to compute with.
These range from various kinds of integers, to fractions, complex
numbers, quaternions, continued fractions, and to numbers represented
with an arbitrary base.

%
\xtcLong{
What is \spad{10} to
the \eth{\spad{100}} power in base \spad{32}?
}{
\spadcommand{radix(10^100,32)}
}{
\[
\begin{array}{@{}l}
\tt
4 I 9 LKIP 9 GRSTC
5 IF 164 PO 5 V 72 ME 827226 JSLAP 46258
5 Q 7 H \\ \tt
00000000000000000000
\end{array}
\]
}{3}{%
RadixExpansion 32%
}


\pseudoSection{Graphics}

You may often want to visualize a symbolic formula or draw
a graph from a set of numerical values.
To do this, you can call upon the \Language{}
graphics capability.

\psXtc{
Draw \texht{$J_0(\sqrt{x^2+y^2})$}{} for \newline
\texht{$-20 \leq x,y \leq 20$}{}.
}{
\spadgraph{draw(5*besselJ(0,sqrt(x**2+y**2)), x=-20..20, y=-20..20)}
}{
\epsffile[0 0 295 295]{../ps/bessintr.ps}
}

Graphs in \Language{} are interactive objects you can manipulate with
your mouse.
Just click on the graph, and a control panel pops up.
Using this mouse and the control panel, you can translate,
rotate, zoom,
change the coloring, lighting, shading, and
perspective on the picture.
You can also generate a PostScript copy of your graph to produce
hard-copy output.

\pseudoSection{\HyperName{}}

\HyperName{} presents you windows on the world of \Language{}, offering
on-line help,
examples, tutorials, a browser, and reference material.
\HyperName{} gives you on-line access to this book in a ``hypertext'' format.
Words that appear in a different font
(for example, \spadtype{Matrix}, \spadfun{factor}, and
\spadgloss{category}) are generally mouse-active; if you click on one
with your mouse, \HyperName{} shows you a new window for that word.

As another example of a \HyperName{} facility,
suppose that you want to compute the roots of
\texht{$x^{49} - 49x^4 + 9$}{} to \smath{49} digits (as in our previous example)
and you don't know how to tell \Language{} to do this.
The ``basic command'' facility of \HyperName{} leads the way.
Through the series of \HyperName{} windows shown in Figure \ref{fig-intro-br}
and the specified mouse clicks, you and \HyperName{}
generate the correct command to issue to compute the answer.


{%
\begin{figure}[thb]
\vspace*{5.10in}
\simpleCaption{Computing the roots of $x^{49}-49x^4+9.$}\label{fig-intro-br}
\vskip .5\baselineskip
\end{figure}
}%

\pseudoSection{Interactive Programming }

\Language{}'s interactive programming language lets you
define your own functions.
A simple example of a user-defined function is one
that computes the successive Legendre polynomials.
\Language{} lets you define these polynomials in a piece-wise way.

%
\xtcLong{
The first Legendre polynomial.
}{
\spadcommand{p(0) == 1}
}{
\vbox{\hbox{\quad}}%
}{}{%
Void%
}
%
\xtcLong{
The second Legendre polynomial.
}{
\spadcommand{p(1) == x}
}{
\vbox{\hbox{\quad}}%
}{}{%
Void%
}
%
\xtcLong{
The \eth{\spad{n}} Legendre polynomial for \texht{$(n > 1)$}{\spad{n > 1}}.
}{
\spadcommand{p(n) == ((2*n-1)*x*p(n-1) - (n-1) * p(n-2))/n}
}{
\vbox{\hbox{\quad}}%
}{}{%
Void%
}

In addition to letting you define simple functions like this,
the interactive language can be used
to create entire application packages.
All the graphs in the \Gallery{} section
in the center of the book, for example,
were created by programs written in the interactive language.

The above definitions for \spad{p} do no computation---they simply
tell \Language{} how to compute \spad{p(k)} for some positive
integer \spad{k}.
To actually get a value of a Legendre polynomial, you ask for it.
\index{Legendre polynomials}

%
\xtcLong{
What is the tenth Legendre polynomial?
}{
\spadcommand{p(10)}
}{
\vbox{\Isize{}%
\vbox{\outputMsg{} Compiling\ function\ p\ with\ type\ Integer\ ->\ Polynomial\ \hfill}
\vbox{\outputMsg{} Fraction\ Integer\ \hfill}
\vbox{\outputMsg{} Compiling\ function\ p\ as\ a\ recurrence\ relation.\hfill}
\vbox{\outputMsg{} \hfill}
\vbox{%
\[
{{{46189}\over{256}}\ {x^{10}}}-{{{109395}\over{256}}\ {x^8}}+
{{{45045}\over{128}}\ {x^6}}-{{{15015}\over{128}}\ {x^4}}+{{{3
465}\over{256}}\ {x^2}}-{{63}\over{256}}
\]
}% end of \vbox
}% end of \vbox
}{7}{%
Polynomial Fraction Integer%
}

\Language{} applies the above pieces for
\spad{p} to obtain the value of \spad{p(10)}.
But it does more:
it creates an optimized, compiled function for \spad{p}.
The function is formed by putting the pieces together into
a single piece of code.
By {\it compiled}, we mean that the function is translated into
basic machine-code.
By {\it optimized}, we mean that certain transformations are
performed on that code to make it run faster.
For \spad{p}, \Language{} actually translates the original definition
that is recursive (one that calls itself)
to one that is iterative (one that consists of a simple loop).

%
\xtcLong{
What is the coefficient of \smath{x^{90}} in \spad{p(90)}?
}{
\spadcommand{coefficient(p(90),x,90)}
}{
\[
{5688265542052017822223458237426581853561497449095175}\over{7
7371252455336267181195264}
\]
}{8}{%
Polynomial Fraction Integer%
}

In general, a user function is type-analyzed and compiled on first use.
Later, if you use it with a different kind of object, the function
is recompiled if necessary.

\pseudoSection{Data Structures}

A variety of data structures are available for interactive use.
These include strings, lists, vectors, sets, multisets, and hash
tables.
A particularly useful structure for interactive use is the
infinite stream:

%
\xtcLong{
Create the infinite stream of derivatives of Legendre
polynomials
}{
\spadcommand{[D(p(i),x) for i in 1..]}
}{
\[
\begin{array}{@{}l}
\displaystyle
\left[ 1, \:{3 \  x}, \:{{{{15}\over 2}\ {x^2}}-{3 \over 2}},
 \:{{{{35}\over 2}\ {x^3}}-{{{15}\over 2}\  x}}, \:{{{{315}\over
8}\ {x^4}}-{{{105}\over 4}\ {x^2}}+{{15}\over 8}}, \: \right.
\\
\\
\displaystyle
\left.{{{{693}\over 8}\ {x^5}}-{{{315}\over 4}\ {x^3}}+{{{105}\over
8}\  x}}, \:{{{{3003}\over{16}}\ {x^6}}-{{{3465}\over{16}}\ {x^
4}}+{{{945}\over{16}}\ {x^2}}-{{35}\over{16}}}, \: \ldots \right]
\end{array}
\]
}{9}{%
Stream Polynomial Fraction Integer%
}

Streams display only a few of their initial elements.
Otherwise, they are ``lazy'': they
only compute elements when you ask for them.

Data structures are an important component for building
application software. Advanced users can
represent data for applications in optimal fashion.
In all, \Language{} offers
over forty kinds of aggregate data structures, ranging
from mutable structures (such as cyclic lists and flexible arrays) to
storage efficient structures (such as bit vectors).
As an example, streams are used as the internal data structure
for power series.

%
\xtcLong{
What is the series expansion
of \texht{$\log(\cot(x))$}{}
about \texht{$x=\pi/2$}{}?
}{
\spadcommand{series(log(cot(x)),x = \%pi/2)}
}{
\[
\begin{array}{@{}l}
\displaystyle
{\log \left({{{2 \  x}- \%pi}\over 2}\right)}+{\log \left({-
1}\right)}+{{1 \over 3}\ {{\left(x -{\%pi \over 2}\right)}^2}}+
{{7 \over{90}}\ {{\left(x -{\%pi \over 2}\right)}^4}}+
\\
\\
\displaystyle
{{{62}\over{2835}}\ {{\left(x -{\%pi \over 2}\right)}^6}}+{O
\left({{\left(x -{\%pi \over 2}\right)}^8}\right)}
\end{array}
\]
}{10}{%
UnivariatePuiseuxSeries(Expression Integer, x, pi/2)%
}

Series and streams make no attempt to compute {\it all} their elements!
Rather, they stand ready to deliver elements on demand.

%
\xtcLong{
What is the coefficient of the \eth{\spad{50}}
term of this series?
}{
\spadcommand{coefficient(\%,50)}
}{
\[
{44590788901016030052447242300856550965644}\over{713146928643
8669111584090881309360354581359130859375}
\]
}{11}{%
Expression Integer%
}

\pseudoSection{Mathematical Structures}

\Language{} also has many kinds of mathematical structures.
These range from simple ones (like polynomials and matrices) to
more esoteric ones (like ideals and Clifford algebras).
Most structures allow the construction of arbitrarily complicated
``types.''

%
\xtcLong{
Even a simple input expression can
result in a type with several levels.
}{
\spadcommand{matrix [[x + \%i,0], [1,-2]]}
}{
\[
\left[
\begin{array}{cc}
{x + \%i}& 0
\\
1 & - 2
\end{array}
\right]
\]
}{12}{%
Matrix Polynomial Complex Integer%
}

The \Language{} interpreter builds types in response to user
input.
Often, the type of the result is changed in order to be applicable
to an operation.

%
\xtcLong{
The inverse operation requires that elements of the above matrices
are fractions.
}{
\spadcommand{inverse(\%)}
}{
\[
\left[
\begin{array}{cc}
{1 \over{x + \%i}}& 0
\\
{1 \over{{2 \  x}+{2 \  \%i}}}& -{1 \over 2}
\end{array}
\right]
\]
}{13}{%
Union(Matrix Fraction Polynomial Complex Integer, ...)%
}

\pseudoSection{Pattern Matching}

A convenient facility for symbolic computation is ``pattern
matching.''
Suppose you have a trigonometric expression and you want to
transform it to some equivalent form.
Use a \spad{rule} command to describe the transformation rules you
\spadkey{rule}
need.
Then give the rules a name and apply that name as a function to
your trigonometric expression.

%
\xtcLong{
Introduce two rewrite rules.
}{
\vtop{
\vbox{\outputMsg{} sinCosExpandRules\ :=\ rule\hfill}
\vbox{\outputMsg{} \ \ sin(x+y)\ ==\ sin(x)*cos(y)\ +\ sin(y)*cos(x)\hfill}
\vbox{\outputMsg{} \ \ cos(x+y)\ ==\ cos(x)*cos(y)\ -\ sin(x)*sin(y)\hfill}
\vbox{\outputMsg{} \ \ sin(2*x)\ ==\ 2*sin(x)*cos(x)\hfill}
\vbox{\outputMsg{} \ \ cos(2*x)\ ==\ cos(x)**2\ -\ sin(x)**2\hfill}
}}{
\[
\begin{array}{@{}l}
\displaystyle
\left\{{{\sin \left({y + x}\right)}= ={{{\cos \left({x}\right)}\ {\sin
\left({y}\right)}}+{{\cos \left({y}\right)}\ {\sin \left({x}\right)}}}},
 \: \right.
\\
\\
\displaystyle
\left.{{\cos \left({y + x}\right)}= ={-{{\sin \left({x}\right)}\ {\sin
\left({y}\right)}}+{{\cos \left({x}\right)}\ {\cos \left({y}\right)}}}},
 \: \right.
\\
\\
\displaystyle
\left.{{\sin \left({2 \  x}\right)}= ={2 \ {\cos \left({x}\right)}\ {\sin
\left({x}\right)}}}, \:{{\cos \left({2 \  x}\right)}= ={-{{\sin
\left({x}\right)}^2}+{{\cos \left({x}\right)}^2}}}\right\}
\end{array}
\]
}{14}{%
Ruleset(Integer, Integer, Expression Integer)%
}

%
\xtcLong{
Apply the rules to a simple trigonometric expression.
}{
\spadcommand{sinCosExpandRules(sin(a+2*b+c))}
}{
\[
\begin{array}{@{}l}
\displaystyle
{{\left(-{{\cos \left({a}\right)}\,{{\sin \left({b}\right)}^2}}-
{2 \,{\cos \left({b}\right)}\,{\sin \left({a}\right)}\,{\sin
\left({b}\right)}}+{{\cos \left({a}\right)}\,{{\cos \left({b}\right)}^
2}}\right)}\,{\sin \left({c}\right)}}-
\\
\\
\displaystyle
{{\cos \left({c}\right)}\,{\sin \left({a}\right)}\,{{\sin \left({b}\right)}^
2}}+{2 \,{\cos \left({a}\right)}\,{\cos \left({b}\right)}\,{\cos
\left({c}\right)}\,{\sin \left({b}\right)}}+
\\
\\
\displaystyle
{{{\cos \left({b}\right)}^2}\,{\cos \left({c}\right)}\,{\sin
\left({a}\right)}}
\end{array}
\]
}{15}{%
Expression Integer%
}

Using input files, you can create your own library of
transformation rules relevant to your applications, then
selectively apply the rules you need.

\pseudoSection{Polymorphic Algorithms}

All components of the \Language{} algebra library are written in
the \Language{} library language.
This language is similar to the interactive language
except for protocols that authors are obliged to follow.
The library language permits you to write ``polymorphic
algorithms,'' algorithms defined to work in
their most natural settings and over a variety of types.

%
\xtcLong{
Define a system of polynomial equations \spad{S}.
}{
\spadcommand{S := [3*x**3 + y + 1 = 0,y**2 = 4]}
}{
\[
\left[{{y +{3 \ {x^3}}+ 1}= 0}, \:{{y^2}= 4}\right]
\]
}{16}{%
List Equation Polynomial Integer%
}
%
\xtcLong{
Solve the system \spad{S} using rational number arithmetic and
30 digits of accuracy.
}{
\spadcommand{solve(S,1/10**30)}
}{
\[
\begin{array}{@{}l}
\displaystyle
\left[{\left[{y = - 2}, \:{x ={{17578796712111842452830704145
07}\over{2535301200456458802993406410752}}}\right]}, \: \right.
\\
\\
\displaystyle
\left.{\left[{y = 2}, \:{x = - 1}\right]}\right]
\end{array}
\]
}{17}{%
List List Equation Polynomial Fraction Integer%
}
%
\xtcLong{
Solve \spad{S} with the solutions expressed in radicals.
}{
\spadcommand{radicalSolve(S)}
}{
\[
\begin{array}{@{}l}
\displaystyle
\left[{\left[{y = 2}, \:{x = - 1}\right]}, \:{\left[{y = 2},
\:{x ={{-{\sqrt{- 3}}+ 1}\over 2}}\right]}, \: \right.
\\
\\
\displaystyle
\left.{\left[{y = 2}, \:{x ={{{\sqrt{- 3}}+ 1}\over 2}}\right]},
 \:{\left[{y = - 2}, \:{x ={1 \over{\root{3}\of{3}}}}\right]},
 \: \right.
\\
\\
\displaystyle
\left.{\left[{y = - 2}, \:{x ={{{{\sqrt{- 1}}\ {\sqrt{3}}}- 1}\over{2
\ {\root{3}\of{3}}}}}\right]}, \:{\left[{y = - 2}, \:{x ={{-{{\sqrt{-
 1}}\ {\sqrt{3}}}- 1}\over{2 \ {\root{3}\of{3}}}}}\right]}\right]
\end{array}
\]
}{18}{%
List List Equation Expression Integer%
}

While these solutions look very different, the results were
produced by the same internal algorithm!
The internal algorithm actually works with equations over any ``field.''
Examples of fields are the rational numbers, floating point
numbers, rational functions, power series, and general expressions
involving radicals.

\pseudoSection{Extensibility}

Users and system developers alike can augment the \Language{}
library, all using one common language.
Library code, like interpreter code, is compiled into machine
binary code for run-time efficiency.

Using this language, you can create new computational types and
new algorithmic packages.
All library code is polymorphic, described in terms of a database
of algebraic properties.
By following the language protocols, there is an automatic,
guaranteed interaction between your code and that of colleagues
and system implementers.
